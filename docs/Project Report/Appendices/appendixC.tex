\chapter{Code Examples}
\begin{comment}
For some projects, it might be relevant to include some code extracts in an appendix. You are not expected to put all of your code here - the correct place for all of your code is in the technical submission that is made in addition to the Project Report. However, if there are some notable aspects of the code that you discuss, including that in an appendix might be useful to make it easier for your readers to access. 

As a general guide, if you are discussing short extracts of code then you are advised to include such code in the body of the report. If there is a longer extract that is relevant, then you might include it as shown in the following section. 

Only include code in the appendix if that code is discussed and referred to in the body of the report. 

\section{Random Number Generator}

The Bayes Durham Shuffle ensures that the psuedo random numbers used in the simulation are further shuffled, ensuring minimal correlation between subsequent random outputs \cite{NumericalRecipes}.

\begin{verbatim}
 code example
\end{verbatim}
\end{comment}

\section{geoTiff - Splitting}
\label{Appendix:code:Splitting}
Split a geotif image into it's individual colour bands.
\begin{verbatim}
def getSplit():
    with TiffFile("images/image-4.tif") as tif:
        image = tif.asarray()
        r = image[:, :, 0]
        g = image[:, :, 1]
        b = image[:, :, 2]
        re = image[:, :, 3]
        ni = image[:, :, 4]

        split = {'red':r,'green':g,'blue':b,'rededge':re,'nearir':ni}
    return split
\end{verbatim}

\section{Averages}
\label{Appendix:code:averages}
Getting the average values for different aprts of an image.
\begin{verbatim}
def getSamples():
    split = getSplit()
    for name, image in split.items():

        image = cv2.normalize(image, None, 0, 255, cv2.NORM_MINMAX, cv2.CV_8U)
        row = image[4]
        print(name)
        #print(row)
        sum = 0
        for x in row:
            sum = sum + x
        backav = sum / len(row)
        print('background', backav)

        sheep = image[27, 25:39]
        #print(sheep)
        sum = 0
        for x in sheep:
            sum = sum + x
        sheepav = sum / len(sheep)
        print('sheep', sheepav)

        print('dif', sheepav - backav)
\end{verbatim}