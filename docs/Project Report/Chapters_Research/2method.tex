%\addcontentsline{toc}{chapter}{Development Process}
\chapter{Experiment Methods}
%TC:ignore
\begin{comment}
This section should discuss the overall hypothesis being tested and justify the approach selected in the context of the research area.  Describe the experiment design that has been selected and how measurements and comparisons of results are to be made. 

You should concentrate on the more important aspects of the method. Present an overview before going into detail. As well as describing the methods adopted, discuss other approaches that were considered. You might also discuss areas that you had to revise after some investigation. 


\end{comment}
%TC:endignore

\section{Hypothesis}

What is a good method to locate the sheep?
Compare how accurate they are.

templating better than thresholding.

\subsection{Best colour band}
When processing images it is difficult to do it on multiple colour bands at the same time, so to determine the best colour band, a simple sample image can be used. In this case the project will use the image in figure \ref{fig:image-4} to help. From this average values can be calculated for each band from pixels with the sheep and pixels without the sheep. A difference can then be calculated to give a measure of how much the sheep stands out form the background.

There is also the option to combine colour bands, that may help emphasis the difference between the foreground and backgrounds. The project will only look at this briefly.

Taking one colour band or merging a combination of bands together gives us a 2d array to work with for each of the methods.

\subsection{Thresholding method}

The first method to be tried is using a threshold value to help us identify the sheep\cite{opencv-t}. 
    
In the example in fig.\ref{thresh} you can see part of the blue band of an image appearing in grayscale on the left, where the brighter pixels of the image correspond to bigger value(255). The central plot shows this in a 3 dimensional graph. You can clearly see where the sheep stands out. 
    
The thresholding method takes a value to threshold for example if the values are greater than 200, it gets set to 255 else to 0. Which produces something like the graph on the right. Then using a simple search for the values of 255 and look at it's neighbouring high values to locate it's centre and estimate it's size.
    
\begin{figure}
    \centering
    \includegraphics[width=5cm]{images/sheep.png}
    \includegraphics[width=5cm]{images/graph.png}
    \includegraphics[width=5cm]{images/graph1.png}
    \caption{Thresholding steps}
    \label{fig:thresh}
\end{figure}

\subsubsection{What is the best threshold value?}

A set of images that range in complexity as seen in Appendix \ref{appendix:Images} can be used to determine the best threshold value to use, by counting the true positives, false positives and false negatives for a range of threshold values. This will give us an idea of what the best threshold value to use is.

\subsection{template matching}

The second method to be tried out is to use a template, either a sheep that we have previously found or an artificial sheep shape to help match a more general sheep shape, ignoring rotation. See fig.\ref{templates} for examples.
    
    \begin{figure}
        \caption{Template Examples}
        \includegraphics[width=4cm]{images/template1.png}
        \includegraphics[width=4cm]{images/template.png}
        \includegraphics[width=4cm]{images/sheep.png}
        \label{templates}
    \end{figure}
    
    The template is then moved across the image and we calculate a metric and get a value to help measure how closely the area matches our template, an example result is shown in fig.\ref{sheeps}, you can see hot spots where the pattern matches most. We can then pick out the spots that have the highest value and these are likely our sheep.
    
    \begin{figure}
    \caption{Templating Search}
        \includegraphics[width=8cm]{images/sheeps.png}
        \includegraphics[width=8cm]{images/sheeps1.png}
        \includegraphics[width=10cm]{images/sheeps2.png}
        \includegraphics[width=8cm]{images/sheeps3.png}
        \label{sheeps}
    \end{figure}
    
\subsubsection{What is the best template to use?}

A set of artificial sheep templates can be generated of different shapes and sizes. Then using the set of images in Appendix \ref{appendix:Images} the number of true positives, false positives and false negatives can be counted and compared to indicate which is the best template to use. 

\subsection{other approaches considered}

Alongside the two more primitive methods considered, there was consideration into trying a machine learning approach, where a object detection model is trained that can find te sheep for us. 
For this a large amount of training and testing data would be required and a significant chunk of time is required to do the training. 
This would be to time consuming for the time scale of this project, as it would likely require training the model a couple of times iterating on the training data set to ensure it is as diverse as possible.

\section{How measurements and comparisons of results are to be made}
A collection of sample images will be used, where the sheep in them can be manually be counted and located this will be the baseline source of truth, then take and process the set of sample images with each method and compare the number of sheep found. 

The issue with comparing the sheep count alone is there be false positives and false negatives giving a false impression that the method is finding the sheep correctly, so we also need to compare where in the images the sheep are detected match the manually found locations. This will show the true positives and allow us to identify the false positives and false negatives.

Using ROC graphs was considered but as there is not fixed number of true negative, the confusion matrices needed would be incomplete. 

\section{Support Tools}
%TC:ignore
\begin{comment}
You should also identify any support tools that you used. You should discuss your choice of implementation tools or simulation tools. For any code that you have written, you can talk about languages and related tools. For any simulation and analysis tools, identify the tools and how they are used on the project. 

For the parts of your project that need some engineering (hardware, software, firmware, or a mixture) to support the experiments, include details in your report about your design and implementation. You should discuss with your supervisor whether it is better to include a different top-level section to describe any engineering work.  In this template, Chapter 3 is suggested as a place for that discussion.
\end{comment}
%TC:endignore

backend: Python3, opencv, matplotlib, numpy, tifffle.
programming UI: object oriented python for ease of use in scripts to build graphs etc.
demo UI: flask simple webserver, django too comlex for this task, bulma make it look nice with not much hasle, (discuus potential to package cross platform with electron). 

design and implementation in Chapter \ref{software}


