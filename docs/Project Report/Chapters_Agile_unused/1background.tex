\chapter{Background \& Objectives}
\begin{comment}
    This section should discuss your preparation for the project, including background reading, your analysis of the problem and the process or method you have followed to help structure your work.  It is likely that you will reuse part of your outline project specification, but at the end of the project you should have more to discuss. 

    \textbf{Note}: 

    \begin{itemize}
        \item All of the sections and text in this example are for illustration purposes. The main Chapters are a good starting point, but the content and actual sections that you include are likely to be different.
   
        \item  Look at the document MMP\_S08 Project Report and Technical Work \cite{ProjectReportTechicalWork} for additional guidance.
    \end{itemize}
\end{comment}

Initial brief:

\begin{quote}
Research projects in collaboration with biologists and farmers look at automatically detecting sheep from aerial imagery, in particular multi-spectral images. This can also applied to other animals such as deer. This project looks into using these images to detect and count sheep. A variety of methods can be explored. 
\end{quote}

The aim of this project is to investigate ways to locate and count the sheep in a set of aerial images of fields, some of which were taken on a red-edge camera to provide multi spectral images. At the start we will mainly look at white sheep, but sheep are not just white so we must investigate ways of identifying black and brown sheep also. I will also discuss how the techniques could be used to count and track other mammals such as deer. The purpose of identifying the animals from the aerial images is that it allows biologists to observe them without disturbing them such as attaching GPS trackers to them individually. The images were originally used by biologists to identify and track the flora but using the same data to track fauna would also advantageous.


\section{Background}
\begin{comment}
What was your background preparation for the project? What similar systems or research techniques did you assess? What was your motivation and interest in this project? 
\end{comment}
The aim of this project is to investigate ways to locate and count the sheep in a set of aerial images of fields, some of which were taken on a red-edge camera to provide multi spectral images. At the start we will mainly look at white sheep, but sheep are not just white so we must investigate ways of identifying black and brown sheep also. I will also discuss how the techniques could be used to count and track other mammals such as deer. The purpose of identifying the animals from the aerial images is that it allows biologists to observe them without disturbing them such as attaching GPS trackers to them individually. The images were originally used by biologists to identify and track the flora but using the same data to track fauna would also advantageous.

I will use these techniques to locate the sheep....

this is the first method, I found some existing research about it....
This is the good things about the first method...
This is the bad things about the first method....
on the whole the first method would be useful in this scenario..

this is the second method, I found some existing research on it
This is the good things about the second method...
This is the bad things about the second method....
on the whole the second method would be useful in this scenario..

this is the third method i found some existing research on it...
This is the good things about the third method...
This is the bad things about the third  method....
on the whole the third method would be useful in this scenario..

This method would be the most superior method to use in comparison to the other methods because....I hope to investigate applying this to the project and proving/disproving my ideas....

\section{Analysis}
\begin{comment}
Taking into account the problem and what you learned from the background work, what was your analysis of the problem? How did your analysis help to decompose the problem into the main tasks that you would undertake? Were there alternative approaches? Why did you choose one approach compared to the alternatives? 

There should be a clear statement of the research questions, which you will evaluate at the end of the work. 

In most cases, the agreed objectives or requirements will be the result of a compromise between what would ideally have been produced and what was felt to be possible in the time available. A discussion of the process of arriving at the final list is usually appropriate.
\end{comment}


\section{Research Method and Software Process}
\begin{comment}
You need to describe briefly the life cycle model or research method that you used. You do not need to write about all of the different process models that you are aware of. Focus on the process model or research method that you have used. It is possible that you needed to adapt an existing method to suit your project; clearly identify what you used and how you adapted it for your needs.

For the research-oriented projects, there needs to be a suitable process for the construction of the software elements that support your work.
\end{comment}


\subsection{Agile Method}
what is it?
why am i using it?

